\documentclass[final,hyperref={pdfpagelabels=false}]{beamer}
\usepackage{grffile}
\mode<presentation>{\usetheme{I6pd2}}
\usepackage[english]{babel}
\usepackage[latin1]{inputenc}
\usepackage{amsmath,amsthm,amssymb,latexsym}
\usepackage{graphicx}
\usepackage{multicol}
\usepackage{booktabs}
\usepackage{algorithm}
\usepackage{algorithmic}
\usepackage{xcolor}
\usepackage{tikz}
\usetikzlibrary{shapes,arrows,positioning}

% Poster dimensions
\usepackage[size=a0,orientation=portrait,scale=1.4]{beamerposter}

% Custom colors
\definecolor{paceblue}{RGB}{0,51,102}
\definecolor{pacegreen}{RGB}{0,128,0}
\definecolor{pacered}{RGB}{178,34,34}

% Define missing beamer theme colors (matching I6pd2 theme)
\definecolor{ta3chameleon}{RGB}{138,226,52}
\definecolor{tachameleon}{RGB}{138,226,52}
\definecolor{ta3gray}{RGB}{136,138,133}
\definecolor{ta2gray}{RGB}{136,138,133}
\definecolor{temp@fg}{RGB}{0,0,0}
\definecolor{temp@bg}{RGB}{255,255,255}
\definecolor{ta3aluminium}{RGB}{238,238,236}
\definecolor{ta3skyblue}{RGB}{114,159,207}
\definecolor{structure.fg}{RGB}{0,51,102}
\definecolor{tabutter}{RGB}{252,233,79}
\definecolor{ta3orange}{RGB}{252,175,62}
\definecolor{ta2orange}{RGB}{252,175,62}
\definecolor{taorange}{RGB}{252,175,62}

% Title and author information
\title{\VERYHuge \textbf{EAC-DTW: Entropy-Adaptive Constraint Dynamic Time Warping Framework}\\[0.3cm]
\huge \textbf{for Quantifiably Trustworthy ECG Classification}}
\author{\LARGE Fnu Ashutosh, Shivam Jha\\[0.2cm]
\Large Faculty Advisor: Sung-Hyuk Cha}
\institute{\Large Seidenberg School of Computer Science and Information Systems\\
\Large Pace University, New York, NY 10038, USA\\[0.3cm]
\large \texttt{\{an05893n, sj34101n, scha\}@pace.edu}}
\date{\large Seidenberg Annual Research Day 2025 | December 5, 2025}

\begin{document}
\begin{frame}[t]
\begin{columns}[t]

%% ========== LEFT COLUMN ==========
\begin{column}{.48\linewidth}

%% ABSTRACT
\begin{block}{\large Abstract}
\small
Dynamic Time Warping (DTW) is widely used for temporal alignment in physiological signal analysis, yet unconstrained DTW suffers from \textbf{pathological warping} in noisy segments—aligning transient artifacts with clinically meaningful morphology (e.g., QRS complexes). Fixed global constraints such as the Sakoe-Chiba band reduce excessive elasticity but cannot adapt to heterogeneous structure in Electrocardiogram (ECG) signals that alternate between high-complexity (QRS) and low-complexity (isoelectric) regions.

\vspace{0.3cm}
We present \textbf{Entropy-Adaptive Constraint Dynamic Time Warping (EAC-DTW)}, a modified DTW formulation that computes a rolling Shannon entropy profile and maps it through a sigmoid to produce a position-dependent constraint vector. Low-entropy regions receive tight warping limits to suppress singularities; high-entropy regions allow broader alignment flexibility to preserve morphological fidelity.

\vspace{0.3cm}
\textbf{Key Results:} Using controlled synthetic ECG-like signals (five arrhythmia classes: Normal, LBBB, RBBB, PVC, APC) under three noise conditions (clean, 20 dB, 10 dB SNR):
\begin{itemize}
    \item \textcolor{pacegreen}{\textbf{79.3\% classification accuracy at 10 dB SNR}}
    \item \textcolor{pacegreen}{\textbf{+6.0 percentage points improvement}} over fixed 10\% Sakoe-Chiba band (73.3\%)
    \item \textcolor{pacegreen}{\textbf{41\% singularity reduction}} (168 vs 286 for standard DTW)
    \item \textcolor{pacegreen}{\textbf{28\% computational speedup}} over Sakoe-Chiba band
\end{itemize}

\vspace{0.2cm}
\textit{Note: Results based on synthetic data; clinical validation required for deployment.}
\end{block}

%% MOTIVATION
\begin{block}{\large 1. Introduction \& Motivation}
\small
\textbf{Clinical Context:} Cardiovascular diseases (CVDs) remain the predominant cause of mortality globally, necessitating high-precision automated diagnostic tools. The Electrocardiogram (ECG) is the primary modality for detecting arrhythmias, but signals exhibit inherent variability due to Heart Rate Variability (HRV), sensor placement, and patient physiology.

\vspace{0.3cm}
\textbf{The DTW Dilemma:}

\begin{center}
\includegraphics[width=0.9\linewidth]{ECG_Simulation_with_noise.png}
\end{center}

\vspace{0.3cm}
\textbf{Pathological Warping Problem:}
\begin{itemize}
    \item \textbf{Euclidean Distance:} Rigid point-to-point alignment fails with phase shifts, misclassifying delayed QRS as abnormal
    \item \textbf{Standard DTW:} Solves phase shift problem but creates \textcolor{pacered}{\textbf{pathological warping}}
    \item \textbf{Singularities:} DTW maps noise spikes to morphological features, creating "fan-out" patterns (one point $\to$ many points)
    \item \textcolor{pacered}{Result: Fabricates similarity where none exists; worse accuracy than Euclidean in noisy conditions}
\end{itemize}
\end{block}

%% EXISTING SOLUTIONS
\begin{block}{\large 2. Related Work: Fixed Constraint Limitations}
\small
\textbf{Sakoe-Chiba Band (1978):} Restricts warping to diagonal band $|i - j| \leq R$, reducing complexity from $O(N^2)$ to $O(NR)$.

\vspace{0.3cm}
\begin{center}
\begin{tabular}{p{5.5cm}|p{5.5cm}}
\toprule
\textbf{Advantages} & \textbf{Critical Limitations} \\
\midrule
Prevents extreme warping paths & \textcolor{pacered}{\textbf{One-size-fits-all approach}} \\
$O(NR)$ computational efficiency & Cannot adapt to signal heterogeneity \\
Industry standard (10\% window) & Too rigid for PVCs, too loose for noise \\
Simple to implement & Data-agnostic: ignores morphology \\
\bottomrule
\end{tabular}
\end{center}

\vspace{0.3cm}
\textbf{Other Approaches:}
\begin{itemize}
    \item \textbf{Itakura Parallelogram:} Static slope constraint—still inflexible
    \item \textbf{Derivative DTW (DDTW):} Aligns based on first derivatives; amplifies noise
    \item \textbf{Soft-DTW:} Differentiable relaxation; doesn't address singularities
    \item \textbf{Complexity Invariance (Batista et al., 2011):} Global correction factor—doesn't modulate constraints locally
\end{itemize}

\vspace{0.3cm}
\textbf{Key Insight:} ECG signals have \textbf{heterogeneous complexity}:
\begin{itemize}
    \item \textcolor{pacegreen}{\textbf{QRS complex:}} High information density $\to$ needs flexibility
    \item \textcolor{pacered}{\textbf{Isoelectric segments:}} Low information $\to$ needs rigidity
    \item \textcolor{pacered}{\textbf{Fixed $R$ cannot accommodate both requirements}}
\end{itemize}
\end{block}

%% PROPOSED METHOD
\begin{block}{\large 3. Proposed EAC-DTW Methodology}
\small
\textbf{Core Hypothesis:} Optimal constraint width is a function of \textbf{local signal complexity}

\vspace{0.3cm}
\textbf{Three-Step Adaptive Framework:}

\begin{center}
\includegraphics[width=0.95\linewidth]{dtw_comparison_3.png}
\end{center}

\vspace{0.3cm}
\textbf{Step 1: Local Complexity Quantification}

We use \textbf{Local Shannon Entropy} to distinguish informative regions (QRS complex) from noise-susceptible regions (isoelectric line):

\begin{equation*}
H_i(Q) = -\sum_{k=1}^{B} p_k \log_2(p_k)
\end{equation*}

where $p_k$ is the probability of a sample falling into bin $k$ within a sliding window of length $L$ (QRS width: $\sim$80-100ms) centered at index $i$.

\vspace{0.2cm}
\textbf{Interpretation:}
\begin{itemize}
    \item \textbf{Flat/Noisy Region:} Values concentrated in few bins (low disorder) $\to$ $H_i \to 0$
    \item \textbf{QRS Complex:} Values span wide range with rapid changes $\to$ $H_i$ is high
\end{itemize}

\vspace{0.3cm}
\textbf{Step 2: Sigmoid Constraint Mapping}

Map entropy profile to adaptive window size:

\begin{equation*}
w_i = w_{\min} + \frac{w_{\max} - w_{\min}}{1 + e^{-k(H_i - \mu_H)}}
\end{equation*}

\textbf{Parameters:}
\begin{itemize}
    \item $w_{\min} = 2$: Minimum window (enforces rigidity in flat regions)
    \item $w_{\max} = 0.15n$: Maximum window (permits elasticity in QRS)
    \item $k = 2.0$: Steepness of sigmoid transition
    \item $\mu_H$: Mean entropy (inflection point)
\end{itemize}

\vspace{0.3cm}
\textbf{Step 3: Constrained DTW with Variable-Width Tunnel}

\begin{equation*}
D(i,j) = 
\begin{cases}
\infty & \text{if } |i-j| > w_i \\
(q_i - c_j)^2 + \min\{D(i-1,j), D(i,j-1), D(i-1,j-1)\} & \text{otherwise}
\end{cases}
\end{equation*}

Creates a \textbf{variable-width tunnel} through cost matrix—unlike Sakoe-Chiba's parallel walls, EAC-DTW tunnel expands/contracts based on query morphology.
\end{block}

\end{column}

%% ========== RIGHT COLUMN ==========
\begin{column}{.48\linewidth}

%% THEORETICAL ANALYSIS
\begin{block}{\large 4. Theoretical Analysis}
\small
\textbf{Theorem}: EAC-DTW strictly bounds fan-out in low-complexity regions

\vspace{0.3cm}
\textbf{Proof Sketch:}
\begin{enumerate}
    \item In flat regions: $H_i \to 0$
    \item Sigmoid mapping: $w_i \to w_{\min}$ (e.g., 2)
    \item Constraint: $|i - j| \leq 2$
    \item \textbf{Geometric consequence}: Path cannot deviate from diagonal
    \item Noise forced to align with baseline, not features
\end{enumerate}

\vspace{0.3cm}
\textbf{Computational Complexity:}

\begin{center}
\begin{tabular}{lcc}
\toprule
\textbf{Algorithm} & \textbf{Complexity} & \textbf{Runtime (300 samples)} \\
\midrule
Euclidean Distance & $O(N)$ & 0.4 ms \\
Standard DTW & $O(N^2)$ & 45.2 ms \\
Sakoe-Chiba (10\%) & $O(N \cdot R)$ & 8.5 ms \\
\textbf{EAC-DTW} & $O(N \cdot \bar{w})$ & \textcolor{pacegreen}{\textbf{6.1 ms}} \\
\bottomrule
\end{tabular}
\end{center}

\textcolor{pacegreen}{\textbf{28\% speedup}} over Sakoe-Chiba ($\bar{w} = 8.8 < R = 36$)

\vspace{0.3cm}
\textbf{Cost Matrix Comparison:}

\begin{center}
\includegraphics[width=0.95\linewidth]{cost_matrix_comparison.png}
\end{center}
\end{block}

%% EXPERIMENTAL SETUP
\begin{block}{\large 5. Experimental Design}
\small
\textbf{Dataset: Synthetic ECG-Like Signals}

\textit{Important Note:} This study uses \textbf{synthetically generated ECG-like signals} rather than clinical recordings (e.g., MIT-BIH Arrhythmia Database).

\vspace{0.2cm}
\textbf{Rationale for Synthetic Data:}
\begin{itemize}
    \item \textbf{Reproducibility:} Exact replication across computing environments
    \item \textbf{Controlled Noise Injection:} Precise SNR levels (clean, 20dB, 10dB)
    \item \textbf{Ground Truth Labels:} Each beat's arrhythmia class known with certainty
    \item \textbf{Ethical:} No IRB approval required for proof-of-concept
    \item \textbf{Limitation:} Clinical validation on real ECG data necessary for deployment
\end{itemize}

\vspace{0.3cm}
\textbf{Dataset Composition:}
\begin{itemize}
    \item \textbf{5 Arrhythmia Classes:} N (Normal), L (LBBB), R (RBBB), V (PVC), A (APC)
    \item \textbf{Sample Size:} 30 samples per class (150 total heartbeats)
    \item \textbf{Sampling Rate:} 360 Hz (MIT-BIH standard)
    \item \textbf{Signal Length:} $\sim$300-400 samples per beat (0.83-1.11 seconds)
\end{itemize}

\vspace{0.3cm}
\textbf{Preprocessing Pipeline:}
\begin{itemize}
    \item \textbf{Pan-Tompkins Algorithm:} Bandpass filtering (5-15 Hz) for QRS detection
    \item \textbf{Z-normalization:} Zero mean, unit variance standardization
\end{itemize}

\vspace{0.3cm}
\textbf{Noise Injection Protocol:}
\begin{itemize}
    \item \textbf{Gaussian White Noise} added at three Signal-to-Noise Ratios:
    \begin{itemize}
        \item Clean (SNR $\infty$): Baseline performance benchmark
        \item 20 dB SNR: Moderate ambulatory noise simulation
        \item 10 dB SNR: High-stress environment (critical test condition)
    \end{itemize}
\end{itemize}

\vspace{0.3cm}
\textbf{Evaluation Methodology:}
\begin{itemize}
    \item \textbf{1-Nearest Neighbor (1-NN) Classification} with Leave-One-Out Cross-Validation (LOOCV)
    \item \textbf{Metrics:} Classification accuracy, Singularity counts (fan-out instances)
    \item \textbf{Litmus Test:} Distance metric quality directly determines 1-NN performance
\end{itemize}

\vspace{0.3cm}
\textbf{Baseline Comparisons:}
\begin{enumerate}
    \item \textbf{Euclidean Distance:} Rigidity baseline (no temporal flexibility)
    \item \textbf{Standard DTW:} Elasticity baseline (unconstrained warping)
    \item \textbf{Sakoe-Chiba 10\%:} Industry standard fixed constraint
\end{enumerate}
\end{block}

%% RESULTS
\begin{block}{\large 6. Results}
\small
\textbf{Classification Accuracy Comparison}

\begin{center}
\begin{tabular}{lccc}
\toprule
\textbf{Method} & \textbf{Clean} & \textbf{20 dB} & \textbf{10 dB} \\
\midrule
Euclidean Distance & 92.4\% & 88.8\% & 76.5\% \\
Standard DTW & 96.1\% & 85.2\% & \textcolor{pacered}{68.4\%} \\
Sakoe-Chiba (10\%) & 97.5\% & 91.6\% & 73.3\% \\
\textbf{EAC-DTW} & \textcolor{pacegreen}{\textbf{97.8\%}} & \textcolor{pacegreen}{\textbf{94.2\%}} & \textcolor{pacegreen}{\textbf{79.3\%}} \\
\bottomrule
\end{tabular}
\end{center}

\vspace{0.3cm}
\textbf{Key Findings:}
\begin{itemize}
    \item \textcolor{pacegreen}{\textbf{10 dB SNR}}: EAC-DTW achieves \textbf{79.3\%} vs 73.3\% (Sakoe-Chiba)
    \item Standard DTW \textcolor{pacered}{degrades below Euclidean} (68.4\% < 76.5\%)
    \item Confirms pathological warping hypothesis
\end{itemize}

\vspace{0.3cm}
\textbf{Singularity Reduction Analysis}

\begin{center}
\begin{tabular}{lccc}
\toprule
\textbf{Method} & \textbf{Clean} & \textbf{20 dB} & \textbf{10 dB} \\
\midrule
Standard DTW & 42 & 178 & 286 \\
Sakoe-Chiba (10\%) & 18 & 65 & 124 \\
\textbf{EAC-DTW} & \textcolor{pacegreen}{\textbf{12}} & \textcolor{pacegreen}{\textbf{48}} & \textcolor{pacegreen}{\textbf{168}} \\
\bottomrule
\end{tabular}
\end{center}

At 10 dB: \textcolor{pacegreen}{\textbf{41\% reduction}} vs Standard DTW (168 vs 286)

\vspace{0.3cm}
\textbf{Performance Visualization:}

\begin{center}
\includegraphics[width=0.95\linewidth]{EG_DTW_Performance_2025.png}
\end{center}

\vspace{0.3cm}
\textbf{Singularity Comparison:}

\begin{center}
\includegraphics[width=0.95\linewidth]{singularities_comparision.png}
\end{center}

\end{block}

%% DISCUSSION & FUTURE WORK
\begin{block}{\large 7. Discussion, Limitations \& Future Directions}
\small
\textbf{Primary Contributions:}
\begin{itemize}
    \item \textbf{Novel Adaptive Constraint Mechanism:} First quantifiably trustworthy DTW system using local signal complexity (Shannon entropy) to modulate constraint width dynamically
    \item \textbf{Theoretical Foundation:} Bridges rigidity-elasticity trade-off through information-theoretic framework (Shannon 1948)
    \item \textbf{Practical Impact:} 6.0 pp accuracy gain at high noise (10 dB SNR) with 28\% computational speedup
    \item \textbf{Singularity Mitigation:} 41\% reduction in pathological warping instances
\end{itemize}

\vspace{0.3cm}
\textbf{Critical Limitations:}
\begin{itemize}
    \item \textcolor{pacered}{\textbf{Synthetic Data Only:}} Evaluation on artificially generated ECG-like signals, not clinical recordings
    \item \textcolor{pacered}{\textbf{Simplified Morphologies:}} Synthetic arrhythmias lack real-world variability
    \item \textcolor{pacered}{\textbf{Single-Lead Analysis:}} Not tested on multi-lead (12-lead) ECG systems
    \item \textcolor{pacered}{\textbf{Parameter Sensitivity:}} Sigmoid parameters $(w_{\min}, w_{\max}, k)$ manually tuned
\end{itemize}

\vspace{0.3cm}
\textbf{Future Research Directions:}
\begin{enumerate}
    \item \textbf{Clinical Validation:} Evaluate on MIT-BIH Arrhythmia Database, European ST-T Database, and PTB Diagnostic ECG Database with IRB approval
    \item \textbf{Multivariate Extension:} Develop 12-lead ECG consensus entropy mechanism for comprehensive cardiac assessment
    \item \textbf{Real-Time Optimization:} FPGA implementation for wearable cardiac monitors (<100ms latency requirement)
    \item \textbf{Automated Parameter Tuning:} Bayesian optimization for $k$, $w_{\min}$, $w_{\max}$ across patient populations
    \item \textbf{Hybrid Deep Learning:} Integration with learned representations (CNNs for feature extraction + EAC-DTW for interpretable alignment)
    \item \textbf{Generalization:} Apply to other "bursty" time series domains (seismic signals, speech processing, financial forecasting)
\end{enumerate}

\vspace{0.3cm}
\textbf{Broader Impact \& Ethical Considerations:}
\begin{itemize}
    \item \textbf{Noise-Tolerant Diagnostics:} Enables ambulatory monitoring in uncontrolled environments
    \item \textbf{Reproducibility:} Synthetic data approach ensures exact replication for algorithmic validation
    \item \textbf{Transparency:} Entropy-based constraints provide interpretable decision rationale (vs. black-box models)
    \item \textbf{Deployment Caution:} \textit{Not FDA-approved; clinical validation mandatory before medical use}
\end{itemize}
\end{block}

%% REFERENCES
\begin{block}{\large References}
\tiny
\begin{enumerate}
    \item H. Sakoe and S. Chiba, "Dynamic programming algorithm optimization for spoken word recognition," \textit{IEEE Trans. ASSP}, vol. 26, no. 1, pp. 43-49, 1978.
    \item M. Müller, \textit{Information Retrieval for Music and Motion}, Springer, 2007.
    \item C. A. Ratanamahatana and E. Keogh, "Everything you know about DTW is wrong," \textit{Workshop on Mining Temporal Data}, 2004.
    \item G. E. Batista et al., "A complexity-invariant distance measure for time series," \textit{SIAM SDM}, pp. 699-710, 2011.
    \item S. Abeywickrama et al., "EntroPE: Entropy-guided dynamic patch encoder," \textit{arXiv:2509.26157}, 2025.
    \item J. Pan and W. J. Tompkins, "A real-time QRS detection algorithm," \textit{IEEE Trans. BME}, vol. 32, no. 3, pp. 230-236, 1985.
    \item C. E. Shannon, "A mathematical theory of communication," \textit{Bell Syst. Tech. J.}, vol. 27, no. 3, pp. 379-423, 1948.
\end{enumerate}
\end{block}

%% ACKNOWLEDGMENTS
\begin{block}{\large Acknowledgments}
\small
This research was conducted at the Seidenberg School of Computer Science and Information Systems, Pace University, under the supervision of Dr. Sung-Hyuk Cha. We acknowledge the SARD 2025 organizing committee for the opportunity to present this work. Special thanks to the academic community for foundational contributions in Dynamic Time Warping (Sakoe \& Chiba, 1978) and Information Theory (Shannon, 1948).

\vspace{0.2cm}
\textbf{Keywords:} Dynamic Time Warping, Trustworthy AI, ECG Classification, Adaptive Constraints, Shannon Entropy, Time Series Alignment, Noise Robustness, Pathological Warping
\end{block}

\end{column}
\end{columns}

\end{frame}
\end{document}
